%**************************************************************************
%* SpringSim 2017 Author Kit
%*
%* Word Processing System: TeXnicCenter and MiKTeX
%*
%**************************************************************************

\documentclass{scspaperproc}

\usepackage{latexsym}
\usepackage{graphicx}
\usepackage{mathptmx}
\usepackage{amsmath}
\usepackage{amsfonts}
\usepackage{amssymb}
\usepackage{amsbsy}
\usepackage{amsthm}
%\usepackage[pdftex,colorlinks=true,urlcolor=blue,citecolor=black,anchorcolor=black,linkcolor=black]{hyperref}
%% \usepackage[dvips,colorlinks=true,urlcolor=blue,citecolor=black,%
%% anchorcolor=black,linkcolor=black]{hyperref}

% custom hyphenation rules
\usepackage{hyphenat}
\hyphenation{op-tical net-works semi-conduc-tor}

% theorem style
\newtheoremstyle{scsthe}% hnamei
{8pt}% hSpace abovei
{8pt}% hSpace belowi
{\it}% hBody fonti
{}% hIndent amounti1
{\bf}% hTheorem head fontbf
{.}% hPunctuation after theorem headi
{.5em}% hSpace after theorem headi2
{}% hTheorem head spec (can be left empty, meaning `normal')i
\theoremstyle{scsthe}
\newtheorem{theorem}{Theorem}
\renewcommand{\thetheorem}{\arabic{theorem}}
\newtheorem{corollary}[theorem]{Corollary}
\renewcommand{\thecorollary}{\arabic{corollary}}
\newtheorem{definition}{Definition}
\renewcommand{\thedefinition}{\arabic{definition}}

% avoid overrunning the right margin
\sloppy

%% ***** NOTE *****

%% The use of the long citation format (e.g. "Brown and Edwards
%% (1993)" rather than "[5]") and at the same time using the hyperref
%% package can lead to hard to trace bugs in case the citation is
%% broken accross the line (usually this will mark the entire
%% paragraph as a hyperlink (clickable) which is easily noticeable and
%% fixed if using colorlinks, but not if the color is black -- as it
%% is now). Worse yet, if a citation spans page boundary, LaTeX
%% compilation can fail, with an obscure error message. Since this
%% depends a lot on the flow of the text and wording, these bugs come
%% and go and can be extremely hard for a beginner to trace. The error
%% message can look like this:
%%
%%    ! pdfTeX error (ext4): \pdfendlink ended up in different nesting
%%    level than \pdfstartlink.  \AtBegShi@Output ...ipout \box
%%    \AtBeginShipoutBox \fi \fi
%%    l.174 
%%    ! ==> Fatal error occurred, no output PDF file produced!
%%
%% and can be universally fixed by putting an \mbox{} around the
%% citation in question (in this case, at line 174) and maybe adapting
%% the wording a little bit to improve the paragraph typesetting,
%% which is perhaps not immediately obvious.
%****************************************************************************

% begin document
\begin{document}

% Page header (author list)
\SCSpagesetup{Lux, Watson, Chang, Bernard, Li, Xu, Back, Butt, Cameron, and Hong}

% Conference info
\def\SCSconferenceacro{SpringSim}
\def\SCSpublicationyear{2018}
\def\SCSconferencedates{April 15-18}
\def\SCSconferencevenue{Baltimore, MD, USA}
\def\SCSsymposiumacro{HPC} % High Performance Computing Symposium

% title
\title{Comparison of Models of I/O Characteristics \\ in High
  Performance Computing Systems}

% AUTHOR LIST
% *** NOTE: May need to adjust titlevboxsize in the preamble
\author{Thomas C. H. Lux \\ [12pt]
Dept. of Computer Science \\
Virginia Polytechnic Institute\\
\& State University \\
Blacksburg, VA 24061 \\
tchlux@vt.edu \\
\and
Layne T. Watson \\[12pt]
Dept. of Computer Science\\
Dept. of Mathematics\\
Dept. of Aerospace \& Ocean Eng.\\ 
Virginia Polytechnic Institute\\
\& State University \\
\and
Tyler H. Chang\\
Jon Bernard\\
Bo Li\\[12pt]
Dept. of Computer Science\\ 
Virginia Polytechnic Institute\\
\& State University \\
\and
Li Xu\\[12pt]
Dept. of Statistics\\ 
Virginia Polytechnic Institute\\
\& State University \\
\and
Godmar Back\\
Ali R. Butt\\
Kirk W. Cameron\\[12pt]
Dept. of Computer Science\\ 
Virginia Polytechnic Institute\\
\& State University \\
\and
Yili Hong\\[12pt]
Dept. of Statistics\\ 
Virginia Polytechnic Institute\\
\& State University \\
}

\maketitle

\section*{Abstract}

Each of high performance computing, cloud computing, and computer
security have their own interests in modeling and predicting the
performance of computers with respect to how they are configured. An
effective model might infer internal mechanics, minimize power
consumption, or maximize computational throughput of a given
system. This paper analyzes a seven-dimensional dataset measuring the
input/output (I/O) characteristics of a cluster of identical computers
using the benchmark IOzone. The I/O performance characteristics are
modelled with respect to system configuration using multivariate
interpolation and approximation techniques. The analysis reveals that
accurate models of I/O characteristics for a computer system may be
created from a small fraction of possible configurations, and that
some modeling techniques will continue to perform well as the number
of system parameters being modeled increases. These results have
strong implications for future predictive analyses based on more
comprehensive sets of system parameters.

\textbf{Keywords:} Regression, approximation, other stuff


%     Introduction     
%======================
\section{Introduction}

Performance tuning is often an experimentally complex and time-intense
chore necessary for configuring HPC systems. The procedures for this
tuning vary largely from system to system and are often subjectively
guided by the system engineer(s). Once a desired level of performance
is achieved, HPC systems often remain in that base configuration with
all subsequent modifications being incremental. The changes made are
often associated with updates and job-specific customizations. In the
case that a system has changing workloads or non-stationary
performance objectives that range from maximizing computational
throughput to minimizing power consumption and system variability, it
becomes obvious that a more effective and automated tool is needed for
configuring systems. This scenario presents a challenging and
important application of multivariate approximation and interpolation
techniques.

\begin{enumerate}
\item The value of multivariate modlling
\item The data context
\item The proposed method for using multivariate models
\item The impact of effective models
\end{enumerate}

\subsection{Approximation}

This paper compares five multivariate approximation techniques that
operate on inputs in $\mathbb{R}^d$ (vectors of $d$ real numbers) and
produce predicted responses in $\mathbb{R}^1$. Three of the techniques
are regression-based and produce models that do not perfectly
reproduce the inputs, but are hopefully more generalizable. The
ramaining two techniques are interpolation techniques that reproduce
input data exactly at the expense of higher model complexity. The
sections below outline the mathematical formulations of each technique
and provide computational complexity bounds.

\subsubsection{Multivariate Adaptive Regression Splines}
\subsubsection{Multi-Layer Perceptron Regressor}
\subsubsection{Support Vector Regressor}

\subsection{Interpolation}
\subsubsection{Linear Shepard}
\subsubsection{Delaunay}


%     Related Work     
%======================
\section{Related Work}
\begin{enumerate}
\item Not sure how much to include here? Shooting for thoroughness or
  simply necessary coverage?
\end{enumerate}


%     Methodology     
%=====================
\section{Methodology}
\subsection{Data}
\subsection{Dimensional Analysis}
\begin{enumerate}
\item Cycling the categorical settings
\item Selecting subsets of 1,2,3 up to 4 dimensions
\item Cycling different training : testing ratios (5:95 $\rightarrow$ 95:5)
\item Generating 200 random training : testing splits that ensure the
  testing points are not outside the convex hull of the training.
\item Selecting training points to be well-spaced using QNSTOPP algorithm.
\end{enumerate}

\subsection{Prediction}
\begin{enumerate}
\item For each file generated from the dimensional analysis, train on
  the training data, evaluate at the testing data points
\end{enumerate}


%     Results     
%=================
\section{Results}
\subsection{I/O Throughput Mean}
\subsection{I/O Throughput Variance}
\subsection{Increasing Dimension}


%     Discussion     
%====================
\section{Discussion}
\subsection{Modelling the System}
\subsection{Quantifying Variability}
\subsection{Extending the Analysis}

%     Future Work     
%=====================
\section{Future Work}


\bibliographystyle{scsproc}
\bibliography{paper}

\end{document}
