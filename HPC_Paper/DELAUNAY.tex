%**************************************************************************
%* SpringSim 2017 Author Kit
%*
%* Word Processing System: TeXnicCenter and MiKTeX
%*
%**************************************************************************

\documentclass{scspaperproc}

\usepackage{latexsym}
\usepackage{graphicx}
\usepackage{mathptmx}
\usepackage{amsmath}
\usepackage{amsfonts}
\usepackage{amssymb}
\usepackage{amsbsy}
\usepackage{amsthm}
%\usepackage[pdftex,colorlinks=true,urlcolor=blue,citecolor=black,anchorcolor=black,linkcolor=black]{hyperref}
%% \usepackage[dvips,colorlinks=true,urlcolor=blue,citecolor=black,%
%% anchorcolor=black,linkcolor=black]{hyperref}

% custom hyphenation rules
\usepackage{hyphenat}
\hyphenation{op-tical net-works semi-conduc-tor}

% theorem style
\newtheoremstyle{scsthe}% hnamei
{8pt}% hSpace abovei
{8pt}% hSpace belowi
{\it}% hBody fonti
{}% hIndent amounti1
{\bf}% hTheorem head fontbf
{.}% hPunctuation after theorem headi
{.5em}% hSpace after theorem headi2
{}% hTheorem head spec (can be left empty, meaning `normal')i
\theoremstyle{scsthe}
\newtheorem{theorem}{Theorem}
\renewcommand{\thetheorem}{\arabic{theorem}}
\newtheorem{corollary}[theorem]{Corollary}
\renewcommand{\thecorollary}{\arabic{corollary}}
\newtheorem{definition}{Definition}
\renewcommand{\thedefinition}{\arabic{definition}}

% avoid overrunning the right margin
\sloppy

%% ***** NOTE *****
%% The use of the long citation format (e.g. "Brown and Edwards (1993)" rather than "[5]") and at the same
%% time using the hyperref package can lead to hard to trace bugs in case the citation is broken accross the
%% line (usually this will mark the entire paragraph as a hyperlink (clickable) which is easily noticeable and fixed
%% if using colorlinks, but not if the color is black -- as it is now). Worse yet, if a citation spans page boundary,
%% LaTeX compilation can fail, with an obscure error message. Since this depends a lot on the flow of the text
%% and wording, these bugs come and go and can be extremely hard for a beginner to trace. The error
%% message can look like this:
%%
%%    ! pdfTeX error (ext4): \pdfendlink ended up in different nesting level than \pdfstartlink.
%%    \AtBegShi@Output ...ipout \box \AtBeginShipoutBox 
%%    \fi \fi 
%%    l.174 
%%    ! ==> Fatal error occurred, no output PDF file produced!
%%
%% and can be universally fixed by putting an \mbox{} around the citation in question (in this case, at line 174)
%% and maybe adapting the wording a little bit to improve the paragraph typesetting, which is perhaps not
%% immediately obvious.
%****************************************************************************

% begin document
\begin{document}

% Page header (author list)
\SCSpagesetup{Chang, Watson, Lux, Bernard, Li, Xu, Back, Butt, Cameron, and Hong}

% Conference info
\def\SCSconferenceacro{SpringSim}
\def\SCSpublicationyear{2018}
\def\SCSconferencedates{April 15-18}
\def\SCSconferencevenue{Baltimore, MD, USA}
\def\SCSsymposiumacro{HPC} % High Performance Computing Symposium

% title
\title{Predicting System Performance by Interpolation using a High-Dimensional
Delaunay Triangulation}

% AUTHOR LIST
% *** NOTE: May need to adjust titlevboxsize in the preamble
\author{Tyler H. Chang \\ [12pt]
Dept. of Computer Science \\
Virginia Polytechnic Institute\\
\& State University \\
Blacksburg, VA 24061 \\
thchang@vt.edu \\
\and
Layne T. Watson \\[12pt]
Dept. of Computer Science\\
Dept. of Mathematics\\
Dept. of Aerospace \& Ocean Eng.\\ 
Virginia Polytechnic Institute\\
\& State University \\
\and
Thomas C. H. Lux\\
Jon Bernard\\
Bo Li\\[12pt]
Dept. of Computer Science\\ 
Virginia Polytechnic Institute\\
\& State University \\
\and
Li Xu\\[12pt]
Dept. of Statistics\\ 
Virginia Polytechnic Institute\\
\& State University \\
\and
Godmar Back\\
Ali R. Butt\\
Kirk W. Cameron\\[12pt]
Dept. of Computer Science\\ 
Virginia Polytechnic Institute\\
\& State University \\
\and
Yili Hong\\[12pt]
Dept. of Statistics\\ 
Virginia Polytechnic Institute\\
\& State University \\
}

\maketitle

\section*{Abstract}

When interpolating computing system performance data, there are many input
parameters that must be considered. 
This requires a model for multivariate interpolation capable of scaling to 
high dimensions.
The Delaunay triangulation is a foundational technique, commonly used to
perform strict interpolation in computer graphics, physics, and geography 
applications.
It has been shown to produce a simplex based mesh with numerous favourable 
properties for interpolation.
While computation of the two- and three-dimensional Delaunay triangulation is 
a well-studied problem, there are numerous technical and theoretical 
limitations to computing a high-dimensional Delaunay triangulation.
In this paper, a new algorithm is proposed for computing interpolated values
from the Delaunay triangulation without computing the complete triangulation.
The proposed algorithm is shown to scale to over 50 dimensions.
Data is presented showing that the Delaunay triangulation greatly outperforms 
several other methods for high-dimensional interpolation on a real world high 
performance computing system problem.

\textbf{Keywords:} Delaunay triangulation, multivariate interpolation,
performance variability, high-dimensional data
%% AUTHOR:
% This is a list of no more than five keywords that will identify your paper in indices and databases (required).
% Do not use the words “computer”, “simulation”, “model”, or “modeling”, since these are all assumed.

\section{Introduction}

\end{document}
