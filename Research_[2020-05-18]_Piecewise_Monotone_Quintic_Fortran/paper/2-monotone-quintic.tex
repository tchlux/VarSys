
\heading{2. MONOTONE QUINTIC INTERPOLATION}

In order to construct a monotone quintic interpolating spline, two
primary problems must be solved. First, reasonable derivative values
at data points need to be estimated. Second, the estimated derivative
values need to be modified to enforce monotonicity on all polynomial
pieces.

% The remainder of this section outlines three algorithms for
% constructing a quintic spline interpolant.  Algorithm 1 uses a
% quadratic facet model to construct initial first and second
% derivative estimates for the interpolating spline, Algorithm 2
% checks monotonicity of an interval, and Algorithm 3 defines a
% procedure for stepping initial first and second derivative estimates
% towards values which result in a monotone quintic spline.

Fritsch and Carlson [1980] originally proposed the use of central
differences to estimate derivatives, however this often leads to extra
and unnecessary {\it wiggles} in the spline when used to approximate
second derivatives. In an attempt to capture the local shape of the
data, this package uses a facet model [Haralick and Watson 1981] to
estimate first and second derivatives at data. Rather than picking a
local linear regressor with minimal residual, this work uses a
quadratic facet model that selects the local quadratic interpolant
with minimum magnitude curvature.

%% ===================================================================
%%                   Algorithm 1: QUADRATIC_FACET
\vskip 5mm
{\parindent 0mm
{\bf Algorithm 1:} {
\tt QUADRATIC\_FACET}$\big(X(1{:}n)$, $Y(1{:}n)$, $i \big)$ \vskip 1mm
where $X_j$,$Y_j$ $\in \bbb{R}$ for $j = 1$, $\ldots$, $n$, and $1 < i
< n$. Returns the slope and curvature at $X_i$ of the local quadratic
interpolant with minimum magnitude curvature.
\vskip 2mm}
{\parindent=6mm
%% -------------------------------------------------------------------
\item{} {\tt if} $\big((Y_i \approx Y_{i-1})$ {\tt or} $(Y_i \approx
  Y_{i+1})\big)$ {\tt then; return $(0,0)$}
\item{} {\tt else if} $\big((Y_{i+1} - Y_i)(Y_i - Y_{i-1}) < 0\big)$ {\tt then}
\itemitem{} The point $(X_i$, $Y_i)$ is an extreme point. The
quadratic with minimum curvature that has slope $0$ at $X_i$ will be returned.
\item{} \codent $f_1 := $ interpolate $(X_{i-1}$,$Y_{i-1})$, $(X_i$,$Y_i)$,
  and $Df_1(X_i) = 0$.
\item{} \codent $f_2 := $ interpolate $(X_i$,$Y_i)$, $(X_{i+1}$,$Y_{i+1})$,
  and $Df_2(X_i) = 0$.
\item{} \codent {\tt if} $\big(|D^2f_1| \leq |D^2f_2|\big)$ {\tt then; return} $\big(Df_1$, $D^2f_1\big)$
\item{} \codent {\tt else;} \hskip 34.25mm {\tt return} $\big(Df_2$, $D^2f_2\big)$
\item{} \codent {\tt endif}
\item{} {\tt else}
\itemitem{} {The point $(X_i$, $Y_i)$ is in a monotone segment of
data. In the following, it is possible that either $f_1$ or $f_3$ does
not exist because $i = 2$ or $i = n-1$. In those cases, the minimum
curvature among existing quadratics is chosen.}
\item{} \codent $f_1 := $ interpolate $(X_{i-2}$,$Y_{i-2})$,
  $(X_{i-1}$,$Y_{i-1})$, and $(X_i$,$Y_i)$.
\item{} \codent $f_2 := $ interpolate $(X_{i-1}$,$Y_{i-1})$,
  $(X_i$,$Y_i)$, and $(X_{i+1}$,$Y_{i+1})$.
\item{} \codent $f_3 := $ interpolate $(X_i$,$Y_i)$,
  $(X_{i+1}$,$Y_{i+1})$, and $(X_{i+2}$,$Y_{i+2})$.
\item{} \codent {\tt if} \hskip 9mm $\big(|D^2f_1| \leq |D^2f_2|, |D^2f_3|\big)$
  {\tt then; return} $\big(Df_1$, $D^2f_1\big)$
\item{} \codent {\tt else if} $\big(|D^2f_2| \leq |D^2f_1|, |D^2f_3|\big)$
  {\tt then; return} $\big(Df_2$, $D^2f_2\big)$
\item{} \codent {\tt else;} \hskip 55mm {\tt return} $\big(Df_3$, $D^2f_3\big)$
\item{} \codent {\tt endif}
\item{} {\tt endif}
}
\vskip 5mm
%% ----------------------------------------------------------------------

The estimated derivative values by the quadratic facet model are not
guaranteed to produce monotone quintic polynomial segments. Ulrich and
Watson [1994] established tight constraints on the monotonicity of a
quintic polynomial piece, while deferring to He{\ss} and Schmidt
[1994] for a relevant simplified case. The following algorithm
implements a sharp check for monotonicity by considering the
nondecreasing case. The nonincreasing case is handled similarly.

%% ===================================================================
%%                    Algorithm 2: IS_MONOTONE
\vskip 5mm
{\parindent 0mm
{\bf Algorithm 2:} {\tt IS\_MONOTONE}$\big(x_0, x_1, f \big)$ \vskip 1mm
where $x_0$, $x_1 \in {\bbb R}$, $x_0 < x_1$, and $f$ is an order six
polynomial defined by $f(x_0)$, $Df(x_0)$, $D^2f(x_0)$, $f(x_1)$,
$Df(x_1)$, $D^2f(x_1)$. Returns {\tt TRUE} if $f$ is monotone
increasing on $[x_0,x_1]$.
\vskip 2mm}
{\parindent=6mm
%% -------------------------------------------------------------------
\item{1.} {\tt if} $\big(f(x_0) \approx f(x_1)\big)$ {\tt then}
\item{2.} \codent {\tt return} $\big( 0 = Df(x_0) = Df(x_1)
  = D^2f(x_0) = D^2f(x_1) \big)$
\item{3.} {\tt endif}
\item{4.} {\tt if} $\big(Df(x_0) < 0$ {\tt or} $Df(x_1) < 0\big)$ {\tt
  then; return FALSE; endif}
\item{5.} $w := x_1 - x_0$
\item{6.} $v := f(x_1) - f(x_0)$ \vskip 1mm
\item{} {The necessity of steps 2 and 4 follows directly from
  the fact that $f$ is $C^2$. The following steps 7--13 coincide with
  a simplified condition for quintic monotonicity that reduces to one
  of cubic positivity studied by Schmidt and He{\ss} [1988]. Given
  $\alpha$, $\beta$, $\gamma$, and $\delta$ as defined by Schmidt and
  He{\ss}, monotonicity results when $\alpha \geq 0$, $\delta \geq 0$,
  $\beta \geq \alpha - 2 \sqrt{\alpha \delta}$, and $\gamma \geq
  \delta - 2 \sqrt{\alpha \delta}$.  Step 4 checked for $\delta < 0$,
  step 8 checks $\alpha < 0$, step 10 checks $\beta < \alpha - 2
  \sqrt{\alpha \delta}$, and step 11 checks $\gamma < \delta - 2
  \sqrt{\alpha \delta}$. If none of the monotonicity conditions were
  violated, then the order six piece is monotone and step 12
  concludes.} \vskip 1mm
\item{7.} {\tt if} $\big(Df(x_0) \approx 0$ {\tt or} $Df(x_1) \approx 0\big)$ {\tt then}
\item{8.} \codent {\tt if} $\big(D^2f(x_1)w > 4Df(x_1)$ {\tt then; return FALSE; endif}
\item{9.} \codent $t := 2 \sqrt{Df(x_0) \left(4Df(x_1) - D^2f(x_1) w \right)}$
\item{10.} \codent {\tt if} $\big(t + 3Df(x_0) + D^2f(x_0)w < 0 \big)$
  {\tt then; return FALSE; endif}
\item{11.} \codent {\tt if} $\big(60v - w\big(24Df(x_0) + 32Df(x_1) - 2t
  + w(3D^2f(x_0) - 5D^2f(x_1))\big) < 0\big)$
\item{}   \codent \codent {\tt then; return FALSE; endif}
\item{12.} \codent {\tt return TRUE}
\item{13.} {\tt endif} \vskip 1mm
\item{} {The following code considers the full quintic monotonicity
  case studied by Ulrich and Watson [1994]. Given $\tau_1$, $\alpha$,
  $\beta$, and $\gamma$ as defined by Ulrich and Watson, a quintic
  piece is proven to be monotone when $\tau_1 > 0$ and $\alpha, \gamma
  \geq -(\beta+2)/2$ when $\beta \leq 6$, and $\alpha, \gamma \geq -2
  \sqrt{\beta-2}$ when $\beta > 6$. Step 12 checks $\tau_1 <= 0$,
  steps 18 and 19 determine monotonicity based on $\alpha$, $\beta$,
  and $\gamma$.} \vskip 1mm
\item{12.} {\tt if} $\big( w\big(2\sqrt{Df(x_0) Df(x_1)} - 3(Df(x_0) +
  Df(x_1))\big) - 24v \leq 0 \big)$
\item{} \codent {\tt then; return FALSE; endif}
\item{14.} $t := \root{3/4}\of{Df(x_0) Df(x_1)}$
\item{15.} $\alpha := (4 Df(x_1) - D^2f(x_1)w) \sqrt{Df(x_0)} / t$
\item{16.} $\gamma := (4 Df(x_0) - D^2f(x_0)w) \sqrt{Df(x_1)} / t$
\item{17.} $\displaystyle \beta := {60v/w + 3\big(w(D^2f(x_1) -
  D^2f(x_0)) - 8(Df(x_0) + Df(x_1))\big) \over 2 \sqrt{Df(x_0) Df(x_1)}}$
\item{18.} {\tt if} $(\beta \leq 6)$ {\tt then; return} $\min(\alpha,\gamma) > - (\beta + 2) / 2$
\item{19.} {\tt else;} \hskip 18mm {\tt return} $\min(\alpha,\gamma) > -2 \sqrt{\beta - 2}$
\item{20.} {\tt endif}
}
\vskip 5mm
%% ----------------------------------------------------------------------

It is shown by Ulrich and Watson [1994] that when $0 = DQ(X_i)$ $=
DQ(X_{i+1})$ $= D^2Q(X_i)$ $= D^2Q(X_{i+1})$, the quintic polynomial
over $[X_i$, $X_{i+1}]$ is guaranteed to be monotone. Using this fact,
the following algorithm shrinks initial derivative estimates until a
monotone spline is achieved and outlines the core routine in the
accompanying package.

%% ===================================================================
%%       Algorithm 3: MQSI - Monotone quintic spline interpolant
\vskip 5mm
{\parindent 0mm
{\bf Algorithm 3:} {\tt MQSI}$\big(X(1{:}n), Y(1{:}n) \big)$ \vskip 1mm
where $(X_i,Y_i) \in (\bbb{R},\bbb{R})$ $i = 1,\ldots,n$ are data
points. Returns monotone quintic spline interpolant $Q(x)$ such that
$Q(X_i) = Y_i$ and is monotone increasing (decreasing) on all
intervals that $Y_i$ is increasing (decreasing).
\vskip 2mm}
{\parindent=6mm
\item{} Approximate first and second derivatives at all $X_i$ with {\tt QUADRATIC\_FACET}.
\item{} {\tt do} $i = 1, \ldots n-1$
\item{} \codent $D Q(X_i)$, $D^2 Q(X_i) :=$ {\tt QUADRATIC\_FACET}$(X$, $Y$, $i)$
\item{} {\tt enddo}
\item{} Identify and store all nonmonotone intervals in a {\tt queue}.
\item{} {\tt do} $i = 1, \ldots n-1$
\item{} \codent {\tt if not IS\_MONOTONE}$(X_i$, $X_{i+1}$, $Q)$ {\tt then}
\item{} \codent \codent Add interval $\big(X_i$, $X_{i+1}\big)$ to {\tt queue}.
\item{} \codent {\tt endif}
\item{} {\tt enddo}
\item{} {\tt do while} $\big($ {\tt queue} of nonmonotone intervals is nonempty $\big)$
\itemitem{} Shrink $DQ$ and $D^2Q$ that border nonmonotone intervals.
\itemitem{} Identify and store all remaining nonmonotone intervals in {\tt queue}.
\item{} {\tt enddo}
}
\vskip 5mm
%% ----------------------------------------------------------------------

Given the minimum curvature nature of the initial derivative
estimates, it is desirable to make the smallest necessary changes to
the initial interpolating spline $Q$ while enforcing monotonicity. In
practice a quasi-bisection search is used in place of solely shrinking
$DQ$ and $D^2Q$, adding an additional step that increases previously
shrunken derivative values by a smaller step. As long as a step size
schedule is set that allows the value $0$ to be obtained in a fixed
number of computations, this has no effect on computational
complexity. Notably, Algorithm 3 is equally capable of ensuring a {\it
  piecewise} monotone quintic spline interpolates piecewise monotone
data.

