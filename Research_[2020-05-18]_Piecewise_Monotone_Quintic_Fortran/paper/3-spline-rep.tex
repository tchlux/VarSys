
\heading{3. SPLINE REPRESENTATION}
The monotone quintic spline interpolant $Q(x)$ is represented in terms of
a B-spline basis. The routine {\tt FIT\_SPLINE} in this package computes
the B-spline coefficients $\alpha_i$ of $Q(x)=\sum_{i=1}^{3n} \alpha_i
B_{i,6,t}(x)$ to match the piecewise quintic polynomial values and (first
two) derivatives at the breakpoints $X_i$, where the spline order is six
and the knot sequence $t$ has the breakpoint multiplicities (6, 3, $\ldots$,
3, 6).  The routine {\tt EVAL\_SPLINE} evaluates a spline represented in
terms of a B-spline basis. A Fortran 2003 implementation {\tt EVAL\_BSPLINE}
of the B-spline recurrence relation evaluation code by C. deBoor [1978]
for the value, derivatives, and integral of a B-spline is also provided.

% This function uses the recurrence relation defining a B-spline:
%
% B_{I,1}(X) = {1, if T(I) <= X < T(I+1),
%              {0, otherwise,
%
% where I is the spline index, J = 2, ..., N-MAX{D,0}-1 is the order, and
%
%                X-T(I)                      T(I+J)-X
% B_{I,J}(X) = ------------- B_{I,J-1}(X) + ------------- B_{I+1,J-1}(X).
%             T(I+J-1)-T(I)                T(I+J)-T(I+1)
%                                                                
% All of the intermediate steps (J) are stored in a single block
% of memory that is reused for each step.
%
% The computation of the integral of the B-spline proceeds from
% the above formula one integration step at a time by adding a
% duplicate of the last knot, raising the order of all
% intermediate B-splines, summing their values, and rescaling
% each sum by the width of the supported interval divided by the
% degree plus the integration coefficient.
%
% For the computation of the derivative of the B-spline, the divided
% difference definition of B_{I,J}(X) is used, building from J = N-D,
% ..., N-1 via
%
%                      (J-1) B_{I,J-1}(X)     (J-1) B_{I+1,J-1}(X)  
% D/DX[B_{I,J}(X)] =  ------------------  -  --------------------.
%                       T(I+J-1) - T(I)         T(I+J) - T(I+1)     
%
% The final B-spline is right continuous and has support over the
% interval [T(1), T(N)).
