\heading{4. COMPLEXITY AND SENSITIVITY}

Both Algorithms 1 and 2 have ${\cal O}(1)$ runtime. Given a fixed schedule
for shrinking derivative values, Algorithm 3 has a ${\cal O}(n)$ runtime
for $n$ data points. In execution, the majority of the time, still ${\cal
O}(n)$, is spent solving the banded linear system of equations for the
B-spline coefficients.  Thus for $n$ data points, the overall execution
time is ${\cal O}(n)$.  The quadratic facet model produces a unique
sensitivity to input perturbation, as small changes in input may cause
different quadratic facets to be associated with a breakpoint, and thus
different initial derivative estimates.  However, the quasi-bisection
search for a point near the monotone boundary in $(\tau_1$, $\alpha$,
$\beta$, $\gamma)$ space usually results in a high quality visually appealing
(meaning less wiggle) monotone quintic spline.  Despite this potential
sensitivity, the quadratic facet model is still preferred because it
generally provides excellent initial estimates of the first and second
derivatives at the breakpoints, with few iterations required to find a
monotone $Q(x)$.

\topinsert
\centerline{\epsfxsize=4truein \epsffile{vis/1-sensitivity.eps}}
{\narrower\noindent\rmVIII Fig.\ 1.
A demonstration of the quadratic facet model's sensitivity to small
data perturbations. In this example two quadratic functions $f_1(x) =
x^2$ over $\{1$,$2$,$5/2\}$ and $f_2(x) = (x-2)^2 + 6$ over
$\{5/2$,$3$,$4\}$ have the same curvature and equal value at
$x=5/2$. Given the five points above Algorithm 2 produces the slope of
the solid blue line because $f_1$ and $f_2$ have equal curvature.
However, perturbing the curvature of $f_2$ down by the machine
precision at values $x=\{3$, $4\}$ causes Algorithm 2 to produce the
slope of the red dashed line at $x = 5/2$. This is the nature of a
facet model and a side effect of associating data with local facets.
\par}
\endinsert
